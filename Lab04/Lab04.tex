\documentclass[a4paper]{article}

\usepackage{mathptmx} % math functions!
\usepackage{amsmath} % moar math
\usepackage[11pt]{moresize} % different letters sizes
\usepackage{float} % enables accurate location of tables
\usepackage[utf8]{inputenc} % so you can type accented letters
\usepackage{caption} % to make personalized captions
\usepackage{graphicx} % required for the inclusion of images
\setlength{\parindent}{1cm} % paragraph indenting
\usepackage[margin=1.45in]{geometry} % margin correction here
\usepackage[brazilian]{babel} %Para traduzir os textos


\title{Máquina de Atwood \\ Experimento 4} % main title
\author{F 229 \\ \textsc{Grupo 1}}
\date{XX de XX, 2014}


\begin{document} % actually starts the document here
\maketitle

% members of the group
\begin{center}
	\begin{tabular}{l r l}
		Integrantes: & Henrique Noronha Facioli & 157986 \\
		& Guilherme Lucas da Silva & 155618 \\
		& Beatriz Sechin Zazulla & 154779 \\
		& Lucas Alves Racoci & 156331 \\
		& Isadora Sophia Garcia Rodopoulos & 158018 \\
	\end{tabular}
\end{center}


\section{Resumo}
Neste experimento, estudamos uma \emph{Máquina de Atwood}, um sistema físico que consiste de: um cilindro de latão funcionando como polia, ou seja, com liberdade de girar em torno de um eixo fixo; um fio que será considerado leve - ou seja, com massa irrelevante -, inestensível - isto é, inelástico; dois corpos (1 e 2) que são pendurados na polia por meio do fio anteriormente citado, onde:
\begin{itemize} 
	\item O corpo 1 consiste de um sub-corpo de massa ${m}_{1}$ e mais $n_{1}$ de $5$ sub-corpos; 
	\item O corpo 2 consiste de um sub-corpo de massa ${m}_{2}$ e mais $n_{2}$ de $5$ sub-corpos; 
	\item Os valores de $n_{1}$ e $n_{2}$ são tais que $n_{1}+n_{2}=5$; 
	\item As massas dos corpos 1 e 2 serão chamadas respectivamente de $m_{1}$ e $m_{2}$.
\end {itemize} 

Sabemos que a diferença entre as massas dos dois corpos gera um torque não nulo na polia, o que nos permite estudar seu Momento de Inércia $I_{0}$ e a aceleração da grávidade $g$, através da fórmula a seguir:

\[
\Delta m=\cfrac{2h}{gR^{2}}(I+MR^{2})\ensuremath{\cfrac{1}{t^{2}}}+\cfrac{\tau_{a}}{gR}
\]


\section{Objetivo}
Este experimento tem como principal objetivo o estudo da máquina de Atwood através da determinação do momento de inércia da polia e do torque da força de atrito, possibilitados a partir da manipulação de um sistema inserido no modelo de estudo.


\section{Procedimentos e coleta de dados}

Na realização deste experimento foram utilizados os seguintes materiais: 
\begin{enumerate} 
	\item Polia de latão com eixo;
	\item Barbante;
	\item Conjunto de discos metálicos;
	\item Trena;
	\item Paquímetro;
	\item Balança de precisão;
	\item Cronômetro.
 \end {enumerate} 

\begin{table}[!ht]
	\begin{center}
		\caption{Modelo de tabela}
		\begin{tabular}{| l | l |}
			\cline{2-2} \multicolumn{0}{c|}{ } & \multicolumn{1}{c|}{\textbf{Massa ($Kg$)}} \\  \cline{1-2}
			\multicolumn{0}{|c|}{\textbf{Medida}} & 1,2790\\ \hline
			\multicolumn{0}{|c|}{\textbf{Erro Instrumental}} & 0,0001\\ \hline
		\end{tabular}
	\end{center}
\end{table}

\begin{table}[!ht]
	\begin{center}
	\caption{Massas dos pesos obtidas experimentalmente}
    \begin{tabular}{|l|l|l|l|l|}
    \hline
    ~ & \textbf{Massa 1} ($kg$) & $\Delta_{Massa 1}$ & \textbf{Massa 2} ($kg$) & $\Delta_{Massa 2}$ \\ \hline
    \textbf{1} & 0,9611  & 0,0002  & 0,8934  & 0,0001  \\ \hline
    \textbf{2} & 0,9513  & 0,0002  & 0,9032  & 0,0001  \\ \hline
    \textbf{3} & 0,942   & 0,0002  & 0,9125  & 0,0001  \\ \hline
    \textbf{4} & 0,9518  & 0,0002  & 0,9027  & 0,0001  \\ \hline
    \textbf{5} & 0,9321  & 0,0002  & 0,9224  & 0,0001  \\ \hline
    \end{tabular}
    \end{center}
\end{table}

\begin{table}[!ht]
	\begin{center}
	\caption{Massa total $M$ do sistema Peso+Discos}
	\begin{tabular}{|l|l|}
	\hline
	\textbf{Massa $M$ ($g$)} & 1854,5\\ \hline
	$\Delta$ & 0,3\\ \hline
	\end{tabular}
	\end{center}
\end{table}

\begin{table}[!ht]
	\begin{center}
	\caption{Altura $h$ da Massa 1 ao chão}
	\begin{tabular}{|l|l|}
	\hline
	Altura ($cm$) & 84,6 \\ \hline
	\end{tabular}
	\end{center}
\end{table}

\begin{table}[!ht]
	\begin{center}
	\caption{Informações da Roldana}
	\begin{tabular}{|l|l|}
	\hline
	Massa Roldana ($g$) & 2053,3 \\ \hline
	Diâmetro da Roldana ($mm$) & 999 \\ \hline
	$\Delta_{Diametro}$ & 0,05	\\ \hline
	\end{tabular}
	\end{center}
\end{table}

\section{Análise dos resultados}

\section{Conclusão}

\end{document}